\documentclass[10pt]{article}
\usepackage[margin=2cm]{geometry}
\usepackage{amssymb}
\usepackage{amsmath}
\usepackage{amsthm}
\usepackage{bm}
\usepackage{caption}
\usepackage[shortlabels]{enumitem}
\usepackage{graphicx}
\usepackage{mathtools}
\usepackage{setspace}
\usepackage{dsfont}
\usepackage{xcolor}
\usepackage{upquote}
\usepackage{booktabs}
\usepackage{multirow}
\usepackage[utf8]{inputenc}
\usepackage[colorlinks=true, linkcolor=red, urlcolor=blue, citecolor=green, linktoc=all]{hyperref}

\usepackage{avant}
\renewcommand{\familydefault}{\sfdefault}


\title{Metapopulation Model}

\begin{document}
\maketitle

The metapopulation model is a combination of a simple SEIR model (without vital dynamics)
combined with metapopulation formulating that follows~\cite{Li489}.
$S$, $E$, $I$ and $R$ have the usual meaning and the index represents region.
$M_{ij}$ is a matrix with entries that give the number of people travelling from region $i$ 
to region $j$ in a timestep (assumed to be constant in time).
We model the dynamics in discrete time as given below.

\begin{align*}
    S_i(t+1) &= S_i(t) -\beta_i(t)  \frac{S_i(t) I_i(t)}{N_i(t)} + \sum_{j} M_{ij} \frac{S_j(t)}{N_j(t)} - \sum_{j} M_{ji} \frac{S_i(t)}{N_i(t)}\\
    E_i(t+1) &= E_i(t) +\beta_i(t)  \frac{S_i(t) I_i(t) }{N_i(t)} - a E_i(t) + \sum_{j} M_{ij} \frac{E_j(t)}{N_j(t)} - \sum_{j} M_{ji} \frac{E_i(t)}{N_i(t)}\\
    I_i(t+1) &= I_i(t) + a E_i(t) - \gamma I_i(t) + \sum_{j} M_{ij} \frac{I_j(t)}{N_j(t)} - \sum_{j} M_{ji} \frac{I_i(t)}{N_i(t)}\\
    R_i(t+1) &= R_i(t) + \gamma I_i(t) + \sum_{j} M_{ij} \frac{R_j(t)}{N_j(t)} - \sum_{j} M_{ji} \frac{R_i(t)}{N_i(t)}\\
    N_i(t+1) &=  \sum_{j} M_{ij} N_j(t) - \sum_{j} M_{ji} N_i(t)\\
\end{align*}
The parameter $\beta_i(t)$ represents a time-varying and localised force of infection,
with $\frac{1}{\beta}$ interpreted as the expected time between contacts.
This model assumes that the incubation period is exponentially distributed
with mean $\frac{1}{a}$.
The parameter $\gamma$ represents the typical length of time for an infected person to 
either recover or die.
In this model, we recover a localised and time-varying reproduction number of
\[
    R_i(t) = \frac{1}{\gamma} \beta_i(t).
\]


\clearpage
\bibliographystyle{apalike}
\bibliography{references}
\end{document}
